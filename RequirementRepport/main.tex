\documentclass{article}
\usepackage[english]{babel}
\usepackage[utf8]{inputenc}
\usepackage{amsmath}
\usepackage{array}
\usepackage{newtxtext}
\usepackage{newtxmath}
\usepackage{color}
\usepackage{textcomp}
\usepackage{listings}
\usepackage{caption}
\usepackage{hyperref}
\usepackage{multicol}
\usepackage{lmodern}
\usepackage{graphicx}
\usepackage{xcolor}
\usepackage{soul}

\usepackage[top=2cm, bottom=2cm, left=2cm, right=2cm]{geometry}

\newcommand{\maintitle}{LINGI2255\\
\vspace{0.5\baselineskip}
Software Development Project\\     % Titre
\vspace{\baselineskip}
{\Large \em Specifications \& Requirements - Team 06}}
\newcommand{\teacher}{Kim \textsc{Mens}} % Professeur
\newcommand{\teachingassistant}{Benoît \textsc{Duhoux}}
\newcommand{\authors}{Youri Mouton (0746-16-00 youri.mouton@student.uclouvain.be),\\ Rémy Voet (6642-16-00 remy.voet@student.uclouvain.be),\\ Nicola Romano(0182-17-00 nicola.romano@student.uclouvain.be),\\ Sophie Madessis (2593-12-00  sophie.madessis@student.uclouvain.be),\\ Samuel Monroe (0467-10-00 samuel.monroe@student.uclouvain.be),\\ Antoine Rime (3624-13-00 antoine.rime@student.uclouvain.be),\\ Ilias Boutchichi (7693-12-00 ilias.boutchichi@student.uclouvain.be),\\ Moers Tristan (2734-16-00
tristan.moers@student.uclouvain.be)} % Auteurs
\newcommand{\HRule}{\rule{\linewidth}{0.5mm}}
\setlength{\parskip}{1ex} % Espace entre les paragraphes

\begin{document}
\newcolumntype{C}{>$c<$}
  % Inspiré de http://en.wikibooks.org/wiki/LaTeX/Title_Creation

\begin{titlepage}

\begin{center}

\begin{minipage}[t]{0.8\textwidth}
  \begin{center}
    \includegraphics [width=70mm]{logo_ucl.jpg} \\[0.5cm]
  \end{center}
\end{minipage}

\vspace{1cm}

\HRule \\[0.4cm]
{\huge \bfseries \maintitle}\\[0.4cm]
\HRule \\[1.5cm]

\begin{minipage}[t]{0.5\textwidth}
  \begin{center} \large
    \emph{Prof.:} \teacher
  \end{center}
\end{minipage}

\vspace{0.5cm}

\begin{minipage}[t]{0.5\textwidth}
  \begin{center} \large
    \emph{TA:} \teachingassistant
  \end{center}
\end{minipage}

\vspace{0.5cm}

\begin{minipage}[t]{\textwidth}
  \begin{center} \large
    \emph{Auteurs:} \authors
  \end{center}
\end{minipage}

\vfill

{\large 2017-2018}

\end{center}

\end{titlepage}
  %\cleardoublepage % Dans le cas du recto verso, ajoute une page blanche si besoin
  %\tableofcontents % Table des matières
  \sloppy          % Justification moins stricte : des mots ne dépasseront pas des paragraphes
  %\cleardoublepage

\section{Module's description}
For the moment the questions are created manually by the professor. To increase the number of questions available for a given skill, we want to give the possibility for a professor to create/solve a large sample of questions based on typical problems.\\
And for the students to efficiently train given skills by exercising on a large set of different problems.\\
It means that first the professor has to create a generic question, for example: $ax^2 + bx + c$ for an equation of the second degree. Then from this generic question we will provide an automatic creation of precise question with  multiple values for a, b and c belonging to natural, integer or rational. We also want to provide some control the domain of the question and answer either integer, rational, complex ...\\

We also need to implement a solver for each type of questions.
The solver should be able to give the students the solution and eventually the resolution step by step of the problem. So that the students knows if he provide the correct answer and can understand his mistake if he has done some. When he is training a given skills.\\

For some problems involving geometrical figures, we would like to implement visualization functionnalities for the student.\\
These functionnalities would take the randomized parameters and display the related figures on the student test.\\
For more advanced shapes or 3D shapes, the visualization functionnalities would provide multiple points of view of these shapes, enabling the best understanding of the shape by the student.\\

%For example let's take the case of the calculation of the perimeter of a circle. Our system will create multiple problems concerning

\section{User stories}

The number before each use story is the value coming from the planning poker session.\\

\begin{enumerate}

    % I want to generate = I want a functionnality to randomly create?

    % As a
    % I want to
    % In order to

    \item 5 :\\ % DONE
    As a teacher\\
    I want to be able to set the domain of a problem\\
    In order to let the question creation functionnalities use this domain for random parameters\\

    \item 13 :\\ % DONE
    As a teacher\\
    I want to have, for each assignement topic, a functionnality to create a related problem based on random values\\
    In order to not having to type specific values each time.\\

    \item 8 :\\ % DONE
    As a teacher\\
    I want a functionnality to create multiple 2nd degree equations from random values to be solved by students\\
    In order to not having to manually type the parameters\\

    \item 3 :\\ % DONE
    As a teacher\\
    I want to be allowed to specify rational, integer or natural answers and parameters for the 2nd degree equations problems\\
    In order to allow the functionnality to use random parameters of different domain types\\

    \item 3 :\\
    As a teacher\\ % DONE
    I want a functionnality to create random triangle on which the student will have to compute the perimeter\\
    In order to not having to manually create a triangle for the exercice\\

    \item 3 :\\
    As a teacher\\ % NEW user story en lien avec la précédente
    I want to be able to choose how the student should resolve a question, if there are multiple ways to find the answer (e.g. find a triangle's perimeter)\\
    In order to evaluate a specific skill\\

    \item 3 :\\
    As a teacher\\ % DONE
    I want a functionnality to create a random triangle on which the student will have to compute the area\\
    In order to not having to manually create a set of triangles\\

    \item 2 :\\
    As a teacher\\ % DONE
    I want a functionnality to create a random circle on which the student will have to compute the area\\
    In order to not having to manually create the circles\\

    \item 2 :\\
    As a teacher\\ % DONE
    I want a functionnality to create a random quadrilateral/rhombus/square/rectangle/parallelogram/trapezium on which the student will have to compute the perimeter\\
    In order to not having to create them manually\\

    \item 2 :\\ % DONE
    As a teacher\\
    I want a functionnality to create a random quadrilateral/rhombus/square/rectangle/parallelogram/trapezium on which the student will have to compute the area\\
    In order to not having to create them manually\\

    \item 3 :\\ % DONE
    As a teacher\\
    I want a functionnality to create a random regular polygon (n=5 to 10) on which the student will have to compute the perimeter\\
    In order to not having to create the polygon manually\\

    \item 3 :\\ % DONE
    As a teacher\\
    I want a functionnality to create a random regular polygon (n=5 to 10) on which the student will have to compute the area\\
    In order to not having to create the polygon manually\\

    \item 2 :\\ % DONE
    As a teacher\\
    I want a functionnality to create a random problem where the student will have to find the length of the side of a triangle (using Pythagoras theorem)\\
    In order to not having to create multiple questions manually \\

    \item 5 :\\ % DONE
    As a teacher\\
    I want a functionnality to create a random statistical data on which the student will have to compute characteristic values (one characteristic value for one question)\\
    In order to not having to create them manually\\

    \item 2 :\\ % DONE
    As a teacher\\
    I want a functionnality to create a random problem where the student will have to compute the simple interest\\
    In order to not having to manually create that kind of problem\\

    \item 3 :\\ % DONE
    As a teacher\\
    I want a functionnality to create a question about compound interest from random values\\
    In order to not having to manually type the parameters of each compound interest problem.\\

    \item 13 : \\ % DONE
    As a teacher\\
    I want to have a functionality to select a kind of object (cylinder, cone, prism, pyramid) and create that object from random values to be included in the assignement


\end{enumerate}

\section{Scenarios}

    \subsection{Assessment for geometry skills}

        \begin{enumerate}

            \item Professor Layton wants to assess his students about basic geometric
            skills.
            \item Layton decides to test his students with perimeter and area
            calculus of triangles (non right-angled).
            \item Layton logs in, reach his class management page and access the
            assessment creation page.
            \item Layton select the skills for area and perimeter calculus of
            non right-angled triangles.
            \item Layton creates the assessment.
            \item Questions have been generated with random values ( not leading
            to extreme shapes of triangle ).
            \item Layton chose to regenerate some of the questions that not
            fit its tastes.
            \item Layton finalize the test creation, his students can now do it online.
        \end{enumerate}

    \subsection{Roots of polynomial assessment}

        \begin{enumerate}
            \item Professor Oak decides that the time has come to test his students
            on determining the roots of a polynomial.
            \item Oak logs in, reach his class management page and access the
            assessment creation page.
            \item Oak select the said skill in the selection option.
            \item Oak creates the assessment.
            \item Oak is now on the assessment modification page.
            \item Oak chose the type of parameters and answer for the questions
            related to root calculus.
            \item Oak generate a polynomial and an expected answer with those
            parameters
            \item Oak validates the test, students can do the assessment online.
        \end{enumerate}
    \subsection{Assessment for students}
    %TODO
    %Split the scenario, one for the Assessment and one for training

        \begin{enumerate}

            \item \hl{Student Sasha has to do the assessment given by teacher Oak.}
            \item \hl{Sasha logs in, reach Assessment section.}
            %\item Sasha chooses a subject that is available.
            \item \hl{Sasha arrives in a Assessment page.}
            \item \hl{Sasha answers at all the exercices given}
            \item \hl{Sasha has finished and leave the assessement.}

            % Only for training, show answers after 3 tries
            %\item Sasha see the correction (not resolution) and read feedback
            %\item Sasha can answers again at the exercice and see again the correction.

        \end{enumerate}

     \subsection{Training for students}
        \begin{enumerate}
            \item \hl{Student Sasha wants to train her skills in a given subject.}

            \item \hl{Sasha logs in, reach the training section}

            \item \hl{Sasha choose one skills amongst those given by her teacher.}

            \item \hl{Sacha arrives in a Training page.}

            \item \hl{Sasha try to answer the question, after 3 tries the correction is given.}

            \item \hl{Sasha decide she is done training and leave the page.}
        \end{enumerate}
\section{Wireframes}

\subsection{Question Configuration}
The professor can choose subject and assessment type or skills. The right side is the assessment type where he can set the domain radius and parameter.\\
\begin{center}
    \includegraphics[scale=0.5]{Question_configuration.png}\\
\end{center}
\subsection{Generated Questions Choice}
The professor can choose questions in the list of generated quesions. There is the possibility to generate 5 new question if some are not suitable.

\begin{center}
    \includegraphics[scale=0.5]{Questions_generated_choice.png}
\end{center}
\subsection{Student Answer}
The student has the problem with an illustration and can answer in the input.\\

\begin{center}
    \includegraphics[scale=0.5]{Student_Answer.png}
\end{center}

\section{Activity diagram}
Red bubbles in the Activity Diagram highlight the main activities we will be working on. Blue arrows represent interaction between student and professor diagrams.
\subsection{Student-System}
\begin{center}
    \includegraphics[height=.8\textheight]{studentAD.png}
\end{center}
\subsection{Professor-System}

\begin{center}
    \includegraphics[height=.8\textheight]{profAD.png}
\end{center}


\section{Planning}
\subsection{Tasks planning}
\begin{center}
    \includegraphics[scale=0.65,angle=90]{multipleproblems.jpg}
\end{center}
\subsection{Delivrables planning}
\begin{itemize}
\item
FRI 29.09
Module to be developed chosen
A detailed ORM or ER model of the structure of the initial database.

\item
FRI 06.10 16:00
First draft of requirements document ready: usage scenarios, user stories and time estimate of user stories.
Final requirements document (10 pages) : usage scenarios, uses cases, mock-up, planning, activity diagram.

\item
FRI 20.10
First draft of design diagrams ready: class and sequence diagrams.

\item
FRI 27.10
Updated requirements document to be delivered by all teams for which the client requested modifications.
Completed version of design document with improved version of class and sequence diagrams + a detailed description of how the module to be developed will integrate with the existing application.
First rough implementation of initial prototype, with corresponding tests.

\item
FRI 03.11
1st prototype ready and working.
Test suites + document explaining the tests.
Document detailing what existing code or parts of the database the 1st prototype has used or touched.

\item
FRI 10.11
1st prototype integrated and merged with main application and prototypes of other teams.

\item
FRI 17.11
Design document with class and sequence diagrams for 2nd prototype + detailed description of how the module to be developed will integrate with the existing application.
First rough implementation of second prototype with corresponding tests.


\item
FRI 24.11
2nd prototype ready and working.
Test suites + document explaining the tests.
Document detailing precisely what existing code or parts of the database the 2nd prototype touched.

\item
FRI 08.12
2nd prototype integrated and merged with the main application and prototypes of other teams
Final technical report
Report on the team organisation

\item
FRI 15.12
Fully documented code.
Final overall version of the tests (unit tests and Selenium), updated according to the two feedbacks.
Final user manual in OpenOffice format, to be handed over to the client (production quality).

\item
MON 18.12
Short video (YouTube link) with a demo of the working prototype.

\item
WED 20.12
Final project defence including a demonstration “in vivo” on the “production” site.



\end{itemize}
\section{Development methodology}

    We will be using SCRUM as the Agile methodology.\\

    SCRUM suits the best our needs for this project for these reasons : \\

        \begin{itemize}
            \item The official planning of the course allows us to organize ourselves
            in multiple sprints, with personal sub-goals to reach the
            official deadline.
            \item Managing a team of eight people might be really difficult using
            Waterfall. With SCRUM, everybody is responsible of his own piece of
            work but also of all the other part of the project.\\
            Nobody's never done with his work and everybody's working towards the
            same goal, which leads to a stronger "team-spirit".
            \item Frequent stand-up meetings are a good source of motivation, a
            useful way to keep everybody's knowledge of the project up-to-date.
            \item The team-leader we elected has already responsibilities about
            communications with the clients and team organization, him being the
            SCRUM-Master won't be an overload of work and will perfectly suits the
            whole organization.

    \end{itemize}

\section{Organisation and distribution of work in the team}

    \begin{itemize}

        \item \textbf{Rémy Voet} : Team-Leader, SCRUM Master
        \item \textbf{Sophie Madessis} : Database Engineer
        \item \textbf{Nicola Romano} : Developer, Test Assistant
        \item \textbf{Youri Mouton} : Lead Developer, Git Master
        \item \textbf{Tristan Moers} : Developer, Team-Leader Assistant
        \item \textbf{Ilias Boutchichi} : Front-End Developer
        \item \textbf{Antoine Rime} : Developer
        \item \textbf{Samuel Monroe} : Behaviour Driven Development, Testing, Front-End Developer
    \end{itemize}

\section{Conclusion}
The course of this first part went well, the scrum methodology is apprehended by everyone. But we encountered difficulties in planning poker our estimates were not close so it took us a long time to discuss. The meeting with the client allowed us to better understand their expectations of the project because it was not well understood. \\

Now with all these points and the client feedback we are ready to start the implementation. We have an idea of what the interface will look like thanks to Wireframes and availables features for user.
\end{document}
